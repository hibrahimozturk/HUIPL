\documentclass[12pt]{article}
\usepackage[english]{babel}
\usepackage{natbib}
\usepackage{url}
\usepackage[utf8x]{inputenc}
\usepackage{amsmath}
\usepackage{graphicx}
\graphicspath{{images/}}
\usepackage{parskip}
\usepackage{fancyhdr}
\usepackage{vmargin}
\usepackage{hyperref}
\usepackage{float}
\usepackage{listings}
%\usepackage[figure]{placeins}
\setmarginsrb{3 cm}{2.5 cm}{3 cm}{2.5 cm}{1 cm}{1.5 cm}{1 cm}{1.5 cm}

\title{Problem Set 5}                             % Title
\author{Halil \.{I}brahim \"{O}zt\"{u}rk \& Furkan Karakuş}               % Author
\date{\today}                                           % Date

\makeatletter
\let\thetitle\@title
\let\theauthor\@author
\let\thedate\@date
\makeatother

\pagestyle{fancy}
\fancyhf{}
\rhead{\theauthor}
\lhead{\thetitle}
\cfoot{\thepage}

\begin{document}

%%%%%%%%%%%%%%%%%%%%%%%%%%%%%%%%%%%%%%%%%%%%%%%%%%%%%%%%%%%%%%%%%%%%%%%%%%%%%%%%%%%%%%%%%

\begin{titlepage}
    \centering
    \vspace*{0.5 cm}
    \includegraphics[scale = 0.5]{hacettepe.jpg}\\[1.0 cm]   % University Logo
    \textsc{\LARGE University of Hacettepe}\\[2.0 cm]   % University Name
    \textsc{\Large BBM 415}\\[0.5 cm]               % Course Code
    \textsc{\large Image Processing Laboratory}\\[0.5 cm]               % Course Name
    \rule{\linewidth}{0.2 mm} \\[0.4 cm]
    { \huge \bfseries \thetitle}\\
    \rule{\linewidth}{0.2 mm} \\[1.5 cm]
    
    \begin{minipage}{0.4\textwidth}
        \begin{flushleft} \large
	   Halil \.{I}brahim \"{O}zt\"{u}rk \\
            21328375                                   % Your Student Number

         \end{flushleft}
            \end{minipage}~
            \begin{minipage}{0.4\textwidth}
         \begin{flushright} \large
            Furkan Karakuş \\
            21228453                                   % Your Student Number

        \end{flushright}
    \end{minipage}\\[2 cm]
    
    {\large \thedate}\\[2 cm]
 
    \vfill
    
\end{titlepage}

%%%%%%%%%%%%%%%%%%%%%%%%%%%%%%%%%%%%%%%%%%%%%%%%%%%%%%%%%%%%%%%%%%%%%%%%%%%%%%%%%%%%%%%%%

\tableofcontents
\pagebreak

%%%%%%%%%%%%%%%%%%%%%%%%%%%%%%%%%%%%%%%%%%%%%%%%%%%%%%%%%%%%%%%%%%%%%%%%%%%%%%%%%%%%%%%%%
\section*{Introduction}
Clustering can be considered the most important unsupervised learning problem; so, as every other problem of this kind, it deals with finding a structure in a collection of unlabeled data.
A loose definition of clustering could be the process of organizing objects into groups whose members are similar in some way.
A cluster is therefore a collection of objects which are similar between them and are dissimilar to the objects belonging to other clusters. Feature is point of interest for image description.


\section{Pixel Level Features}

Two different features for each pixel in the image obtained; RGB colors feature and spatial location feature in a vector like [R G B x y]. It was used for the K-means clustering. Color and location values have different range of numbers.Special attention was paid for it.

\section{Superpixel Level Features}
\subsection{Problem Definition}
In this part we introduce a novel algorithm called SLIC (Simple Linear Iterative Clustering) that clusters pixels in the combined three-dimensional color and image plane space to efficiently generate superpixels. 

\subsection{Solution}
\subsubsection{Steps}



\subsubsection{Inputs Outputs}


\subsection{Effects of Parameters}
\paragraph{•}

\section{K-means Clustering}
K-means clustering is the easiest way to fix clustering problem. The main idea is to define k field, one for each cluster. 
\subsection*{Steps for K-means Clustering}
\begin{enumerate}
\item Randomly place K points into the space represented by the objects that are being clustered. These points represent initial group fields.
\item Assign each object to the group that has the closest field.
\item When all objects have been assigned, recalculate the positions of the K field.
\item Repeat Steps 2 and 3 until no more longer move..
\end{enumerate}
\subsection{Solution}


\subsubsection{Inputs Outputs}


\subsection{Effects of Parameters}
\paragraph{•}

\bibliographystyle{plain}
\bibliography{biblist}
  \begin{thebibliography}{1}
  \bibitem{1} \url{https://en.wikipedia.org/wiki/Pyramid_(image_processing)}
  \end{thebibliography}             
\end{document}


